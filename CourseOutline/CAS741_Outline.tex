\documentclass[12pt]{article}

\usepackage{hyperref}

\hypersetup{
    bookmarks=true,         % show bookmarks bar?
      colorlinks=true,       % false: boxed links; true: colored links
    linkcolor=red,          % color of internal links (change box color with linkbordercolor)
    citecolor=green,        % color of links to bibliography
    filecolor=magenta,      % color of file links
    urlcolor=cyan           % color of external links
}

\oddsidemargin 0mm
\evensidemargin 0mm
\textwidth 165mm
\textheight 205mm

\title{Computing and Software 741 (Computational Engineering and Science 741)
  Development of Scientific Computing Software}
%\author{Spencer Smith}

\pagestyle {plain}
\pagenumbering {arabic}

\newcommand{\be}{\begin{enumerate}}
\newcommand{\ee}{\end{enumerate}}
\newcommand{\bi}{\begin{itemize}}
\newcommand{\ei}{\end{itemize}}

% enumi counter is redefined for the purpose of numbering the deliverables
\renewcommand{\labelenumi}{\arabic{enumi}.}
\newcounter{temp}

\begin {document}

\maketitle

This course outline contains important information that will effect your
grade.  You should retain and refer to this outline throughout the term.

\section {Instructor}

Dr.~Spencer Smith\\
Office: ITB/167\\
E-mail: \texttt{smiths@mcmaster.ca}\\
Web: \url{http://www.cas.mcmaster.ca/~smiths}\\
Office Hours: Drop in, or by appointment\\

\section{Calendar Description} 

This course presents the basic principles of software development for reliable
and sustainable scientific and engineering software.  Using example
applications, a systematic process is given for the development and
documentation of requirements, system design, module design, implementation,
testing and inspection.

\section{Introduction}

Scientific computation consists of using computer tools to simulate mathematical
models of real world systems so that we can better understand and predict the
system's behaviour.  A small sample of some important applications of scientific
computation include the following: designing new automotive parts, analysing the
flow of blood in the body, and determining the concentration of a pollutant
released into the groundwater.  As these examples illustrate, scientific
computation can be used for tackling problems that impact such areas as
manufacturing, financial planning, environmental policy, and the health, welfare
and safety of communities.  Given the important applications of scientific
computation, it is surprising that little emphasis is currently placed on the
quality of the software that performs the computations.  Although many
successful and sophisticated algorithms have been developed for scientific
computing, the software often suffers from problems with such qualities as
reliability, usability, verifiability, maintainability, reusability and
portability.  This is why scientific software is routinely sold with a
disclaimer instead of a warranty.  To make matters worse, the quality of
scientific software is becoming increasingly more of an issue because the
complexity and size of the problems that can be simulated on modern computers is
constantly growing.  The question for the future is how to meet the growing need
for providing quick solutions to large and complex problems, and at the same
time ensure that the solutions are correct?  This graduate course will
investigate this question by applying to scientific computing problems such
software engineering methodologies as commonality analysis, requirements
analysis and documentation, modular decomposition, module interface
specification, testing, code and document generation and assurance cases.  

The course will look at tools, techniques and principles for iterative and
incremental development of scientific and engineering software.  Despite the
iterative development cycle, the documentation will follow 5 rational steps: i)
identify the problem, ii) document the requirements, iii) design the system, iv)
implement the software, and v) perform tests.  This structure is well suited to
scientific computing because it parallels the idealized scientific method, as
follows: i) a physical problem of engineering or scientific importance is
identified; ii) a system of governing equations and the associated boundary
conditions are derived; iii) a numerical algorithms are developed; iv) the
numerical algorithms are implemented on a computer; and, v) the model and the
computed results are verified and validated, with the potential to return to one
of the previous steps if necessary.  These five steps are inherently
multidisciplinary as they involve skills from physical modelling, mathematics,
numerical analysis and computer science.  For this reason it is important that
requirements (including assumptions) and design decisions are clearly
documented.

\subsection*{Course Web Site}

This course will be administered via Avenue to Learn.  Go to

\bi

  \item[] \url{http://avenue.mcmaster.ca/} 

\ei

\noindent to access the course's Avenue to Learn page.  Please send only normal 
McMaster e-mail; do not send mail via Avenue.

Students should be aware that, when they access the electronic
components of this course, private information such as first and last
names, user names for the McMaster e-mail accounts, and program
affiliation may become apparent to all other students in the same
course.  The available information is dependent on the technology
used.  Continuation in this course will be deemed consent to this
disclosure.  If you have any questions or concerns about such
disclosure please discuss this with the Instructor.

\emph{It is the student's responsibility to be aware of the
  information on the course's Avenue to Learn page and to check
  regularly for announcements.}
  
The primary purpose of Avenue will be for maintaining grades.  Most
of the course content will be maintained in a public git repository.
You can access this repository at:\\

\href{https://gitlab.cas.mcmaster.ca/smiths/cas741/}{https://gitlab.cas.mcmaster.ca/smiths/cas741/}\\

\noindent Rather than use the Avenue discussion board, please post your
questions (issues) to the GitLab issue tracker.

In addition to Avenue and the Gitlab course note repository, every student will
create a public gitHub repository (with the instructor added as a full access
collaborator) for their work.  The GitHub server is located at
\url{https://github.com/}.  Students will be expected to use GitHub to provide
comments on the work of other students in the class.

\section {Course Project}

At the beginning of the term each student will select a scientific computing
problem.  Over the course of the term software will be developed to address the
selected problem.  The software development process will follow the iterative
waterfall model, with the following milestones:

\begin {enumerate}

\item Software Requirements Specification (SRS)
\item Module Guide (MG)
\item Module Interface Specification (MIS)
\item Implementation (any appropriate programming language)
\item Verification and Validation (VnV) Plan (divided between system and unit documents)
\item Verification and Validation (VnV) Report (divided between system and unit documents) 

\end {enumerate}

With approval from the instructor, the deliverables can potentially be modified,
if a project is more suited to a different structure.  For instance, a project
could replace one of the above deliverables with an assurance case deliverable,
or with domain specific code to automatically build the deliverables.

\section {Course Structure}

The format of the course will consist of student and instructor presentations.
Each student will be expected to do an informal presentation on some subset of their SRS, MG,
MIS, Implementation, VnV Plan and VnV Report.  It is expected the class discussion will
assist in improving the quality of the written deliverables.  Each student will
be expected to hand in the following written documents: SRS, MG, MIS, VnV Plan,
VnV Report, code and Final Documentation.

\section {Grading}

\begin {enumerate}

\item Presentations and class discussion 5\%

\item ``Domain Expert'' and secondary reviewer roles 10\%

\item Problem Statement 0\%

\item System Requirements Specification (SRS) 15\%

\item System Verification and Validation (VnV-Syst) Plan 15\%

\item Module Guide and Module Interface Specification (MG and MIS) 15\%

\item Final Documentation 40\%
\begin{enumerate}
\item Problem Statement (Revised)
\item SRS (Revised)
\item System VnV Plan (Revised)
\item MG (Revised)
\item MIS (Revised)
\item Unit VnV Plan
\item Code
\item System VnV Report
\item Unit VnV Report
\item Reflection Report
\end{enumerate}

\item Drasil simple physics example, pull request accepted 5\% (Bonus)

\end {enumerate}

Each student will be assigned as a ``Domain Expert'' (DE) for one of their
classmates.  The DE will review (via GitHub issues) their assigned project.  In
addition, they will write the code for one of the modules of their assigned
project and issue a pull request through GitHub.  For the secondary reviewer
role, the instructor will assign various review tasks throughout the term.  The
secondary reviewers feedback will focus more on the structure and format of the
documentation, rather than on issues related to domain knowledge.

The potential bonus marks involve creating a simple physics example in Drasil
(\url{https://github.com/JacquesCarette/Drasil}).  Drasil represents recent
(still incomplete) work to automate many of the tasks that are currently done
manually for design and documenting SC software.

\section {Policy Statements}

This section of the course outline explains the course policy with respect to
improving the course, missed work, discrimination and academic dishonesty.

\subsection {Improving the course}

Ideas to improve the course are always welcome.  Moreover, if you have problems
in the course please contact the instructors as early as possible.

\subsection {Missed Work}

Late assignments, without permission, will receive a penalty of -20 \% per
working day that the assignment is late.

\subsection {Discrimination}

The Faculty of Engineering is concerned with ensuring an environment
that is free of all adverse discrimination.  If there is a problem,
that cannot be resolved by discussion among the persons concerned,
individuals are reminded that they should contact their Department
Chair and the Human Rights and Equity Services (HRES) office as soon
as possible.

\subsection {Academic Dishonesty}

You are expected to exhibit honesty and use ethical behaviour in all aspects of
the learning process.  Academic credentials you earn are rooted in principles of
honesty and academic integrity.

Academic dishonesty is to knowingly act or fail to act in a way that results or
could result in unearned academic credit or advantage.  This behaviour can result
in serious consequences, e.g., the grade of zero on an assignment, loss of
credit with a notation on the transcript (notation reads: ``Grade of F assigned
for academic dishonesty''), and/or suspension or expulsion from the university.

It is your responsibility to understand what constitutes academic dishonesty.
For information on the various types of academic dishonesty please refer to the
Academic Integrity Policy, located at
\[\texttt{http://www.mcmaster.ca/academicintegrity/}\]

\noindent The following illustrates only three forms of academic dishonesty:

\be

\item Plagiarism, e.g., the submission of work that is not one's own or for
  which other credit has been obtained.

\item Improper collaboration in group work.

\item Copying or using unauthorized aids in tests and examinations.

  \ee

\emph{Your work must be your own.}  Plagiarism and copying will not be
tolerated!  If it is discovered that you plagiarized or copied, it
will be considered as academic dishonesty.

Students may be asked to defend their written work orally.

\subsection*{Course Modifications}

The instructor and university reserve the right to modify elements of
the course during the term.  The university may change the dates and
deadlines for any or all courses in extreme circumstances.  If either
type of modification becomes necessary, reasonable notice and
communication with the students will be given with explanation and the
opportunity to comment on changes. It is the responsibility of the
student to check their McMaster e-mail and course websites weekly
during the term and to note any changes.  Your McMaster e-mail is the
one with the address ending in \texttt{@mcmaster.ca}.  This is a
separate e-mail address from your Avenue address.

\subsection*{Conduct Expectations}

As a McMaster graduate student, you have the right to experience, and the
responsibility to demonstrate, respectful and dignified interactions within all
of our living, learning and working communities. These expectations are
described in the Code of Student Rights \& Responsibilities (the ``Code''). All
students share the responsibility of maintaining a positive environment for the
academic and personal growth of all McMaster community members, whether in
person or online.

It is essential that students be mindful of their interactions online, as the
Code remains in effect in virtual learning environments. The Code applies to any
interactions that adversely affect, disrupt, or interfere with reasonable
participation in University activities. Student disruptions or behaviours that
interfere with university functions on online platforms (e.g. use of Avenue 2
Learn, WebEx or Zoom for delivery), will be taken very seriously and will be
investigated. Outcomes may include restriction or removal of the involved
students’ access to these platforms.

\subsection*{Academic Accommodation of Students with Disabilities}

Students with disabilities who require academic accommodation must contact
Student Accessibility Services (SAS) at 905-525-9140 ext. 28652 or
sas@mcmaster.ca to make arrangements with a Program Coordinator. For further
information, consult McMaster University’s Academic Accommodation of Students
with Disabilities policy. 

\subsection*{Academic Accommodation for Religious, Indigenous or Spiritual
  Observations (RISO)}

Students requiring academic accommodation based on religious, indigenous or
spiritual observances should follow the procedures set out in the RISO
policy. Students should submit their request to their Faculty Office normally
within 10 working days of the beginning of term in which they anticipate a need
for accommodation or to the Registrar's Office prior to their
examinations. Students should also contact their instructors as soon as possible
to make alternative arrangements for classes, assignments, and tests.

\subsection*{Copyright and Recording}

Students are advised that lectures, demonstrations, performances, and any other
course material provided by an instructor include copyright protected works. The
Copyright Act and copyright law protect every original literary, dramatic,
musical and artistic work, including lectures by University instructors

The recording of lectures, tutorials, or other methods of instruction may occur
during a course. Recording may be done by either the instructor for the purpose
of authorized distribution, or by a student for the purpose of personal
study. Students should be aware that their voice and/or image may be recorded by
others during the class. Please speak with the instructor if this is a concern
for you.

\subsection*{Extreme Circumstances} 

The University reserves the right to change the dates and deadlines for any or
all courses in extreme circumstances (e.g., severe weather, labour disruptions,
etc.). Changes will be communicated through regular McMaster communication
channels, such as McMaster Daily News, A2L and/or McMaster email.

\end{document}