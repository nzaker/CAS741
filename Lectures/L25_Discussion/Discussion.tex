%\documentclass[handout]{beamer} 
\documentclass[t,12pt,numbers,fleqn]{beamer}
%\documentclass[ignorenonframetext]{beamer}

\newif\ifquestions
%\questionstrue
\questionsfalse

\usepackage{pgfpages} 
\usepackage{hyperref}
\hypersetup{colorlinks=true,
    linkcolor=blue,
    citecolor=blue,
    filecolor=blue,
    urlcolor=blue,
    unicode=false}
\urlstyle{same}

\bibliographystyle{plain}

%\usetheme{Iimenau}

\useoutertheme{split} %so the footline can be seen, without needing pgfpages

%\pgfpagesuselayout{resize to}[letterpaper,border shrink=5mm,landscape]  %if this is uncommented, the hyperref links do not work

\mode<presentation>{}

\input{../def-beamer}

\newcommand{\topic}{25 Discussion}

%Title page information for 1D04 lectures slides

% Define year specific parameters - used in title page and footer

\newcommand{\season}{Fall} %use to switch between Winter and Fall
\newcommand{\instructor}{Dr.~Spencer Smith} %use to switch instructor
\newcommand{\instructSmall}{Dr.~Smith}
\newcommand{\yr}{2019}
\newcommand{\courseCode}{CAS 741, CES 741}
\newcommand{\courseTitle}{Development of Scientific Computing Software}

%\setbeamerfont{structure}{series=\bfseries}
%\usefonttheme[stillsansseriftext,stillsansserifmath]{serif}
\setbeamertemplate{navigation symbols}{} 
\setbeamertemplate{itemize item}[ball]

\title{
  {\normalsize \bf 
    \borange{\courseCode~(\courseTitle)\\ \season~\yr}}\\[2ex]
  {\Large \bf \topic}}

\author[Smith]{\instructor}

\institute{
  Faculty of Engineering,
  McMaster University}

\date{
\today
%January 2011\\
\bc
  \includegraphics[scale = 0.2, keepaspectratio]
  {../mcmaster-logo-full-color.jpg}
\ec
}

\renewcommand{\borange}[1] %orange is too hard to read
{
   \bred{#1}
}

\begin{document}

\input{../footline}

%%%%%%%%%%%%%%%%%%%%%%%%%%%%%%%%%%%%%%%%%%%%%%%%%%%%%%

\begin{frame}
\frametitle{Discussions}

\bi
\item Administrative details
\item Coding style
\item Test report
\item Questions?
\item Discussion
\ei

\end{frame}

%%%%%%%%%%%%%%%%%%%%%%%%%%%%%%%%%%%%%%%%%%%%%%%%%%%%%%

\begin{frame}
\frametitle{Administrative Details}

\bi
\item Course evaluation
\bi
\item Nov 23 to Dec 7
\item \url{https://evals.mcmaster.ca}
\ei
\item GitHub issues for colleagues
\bi
\item Assigned 1 colleague (see \texttt{Repos.xlsx} in repo)
\item Provide at least 5 issues on their MIS
\item Grading as before
\item Due by Tuesday, Dec 5, 11:59 pm
\ei
\item GitHub issues for implementation
\bi
\item Not required as part of course
\item Will assign names anyway
\ei
\item Source in src folder
\item Added an INSTALL.txt file to BlankProjectTemplate
\ei

\end{frame}

%%%%%%%%%%%%%%%%%%%%%%%%%%%%%%%%%%%%%%%%%%%%%%%%%%%%%%

\begin{frame}
\frametitle{Administrative Details: Deadlines}
~\newline
\begin{tabular}{l l}
\textbf{Final Documentation} & Dec 18, 11:59 pm\\
\end {tabular}

\end{frame}

%%%%%%%%%%%%%%%%%%%%%%%%%%%%%%%%%%%%%%%%%%%%%%%%%%%%%%

\begin{frame}
\frametitle{Coding Style}

\bi
\item Having a coding standard is more important than which standard you use
\item Examples
\bi
\item Google guides
\bi
\item \href{https://google.github.io/styleguide/pyguide.html}{Python}
\item \href{https://google.github.io/styleguide/cppguide.html}{C++}
\item \href{https://google.github.io/styleguide/javaguide.html}{Java}
\ei
\item
  \href{https://developer.mozilla.org/en-US/docs/Mozilla/Developer_guide/Coding_Style}{Mozilla
    Developer Network}
\item \href{http://homepages.inf.ed.ac.uk/dts/pm/Papers/nasa-c-style.pdf}{NASA C
    Style Guide}
\ei
\item Your decisions on style may evolve over the project
\item Important to be consistent
\ei

\end{frame}

%%%%%%%%%%%%%%%%%%%%%%%%%%%%%%%%%%%%%%%%%%%%%%%%%%%%%%%%%%%%

\begin{frame}
\frametitle{Installability and Learnability}

\bi
\item You can test this
\item Ask a colleague to install your software
\item Run it on a virtual machine, like
  \href{https://www.virtualbox.org/}{VirtualBox}
\item Use a ``light weight'' VM like docker
\item Include installation instructions (INSTALL.txt)
\item Include instructions so that someone else can run your tests cases
\ei

\end{frame}

%%%%%%%%%%%%%%%%%%%%%%%%%%%%%%%%%%%%%%%%%%%%%%%%%%%%%%%%%%%%

\begin{frame}
\frametitle{Final Documentation: Test Report}
\begin{itemize}
\item Completing what you proposed in your test plan
\item You do not need to repeat material from your test plan - the emphasis is
  not on the rational for test case selection, but on the results.
\item If your test plan does not match what you are now testing, edit your test
  plan to ``fake'' a rational design process.
\end{itemize}
\end{frame}

%%%%%%%%%%%%%%%%%%%%%%%%%%%%%%%%%%%%%%%%%%%%%%%%%%%%%%

\begin{frame}
\frametitle{Test Report Continued}
\begin{itemize}
\item Provide specific test cases
\item Summarize your test results
\bi
\item Test case name
\item Initial state
\item Input
\item Expected results
\item Whether actual output matched expected
\ei
\item Summarize and explain usability tests - quantify the results
\item Performance tests - quantify the results
\item Stress tests
\item Robustness tests
\item After quantification of nonfunctional tests, explain significance of
  results
\end{itemize}
\end{frame}

%%%%%%%%%%%%%%%%%%%%%%%%%%%%%%%%%%%%%%%%%%%%%%%%%%%%%%

\begin{frame}
\frametitle{Test Report Continued}
\begin{itemize}
\item In cases where there are many similar tests
\bi
\item Summarize the results
\item If the expected result is obvious, you might not need to state it
\item Give an example test case, and explain how similar tests were constructed
\item If the tests were random, describe how they were selected, and how many,
  but not all of the details
\item Use graphs and tables
\item You need enough information that
\bi
\item Someone could reproduce your tests
\item Your test results are convincing
\item Evidence that you have used testing to improve the quality of your project
\ei
\ei
\end{itemize}
\end{frame}

%%%%%%%%%%%%%%%%%%%%%%%%%%%%%%%%%%%%%%%%%%%%%%%%%%%%%%

\begin{frame}
\frametitle{Test Report Continued}
\begin{itemize}
\item Summarize changes made in response to test results
\item Explain your automated testing set-up (if require more detail than from
  the test plan)
\item Provide traceability to requirements (if not in test plan)
\item Provide traceability to modules (if not in test plan)
\item Make sure you show test results for ``bad/abnormal'' input
\end{itemize}
\end{frame}

%%%%%%%%%%%%%%%%%%%%%%%%%%%%%%%%%%%%%%%%%%%%%%%%%%%%%%

\begin{frame}
\frametitle{Sample Test Report Documents} 

\bi
\item
  \href{https://gitlab.cas.mcmaster.ca/screenholders/screenholders}{Screenholders}
\item
  \href{https://github.com/palmerst/CS-4ZP6/tree/master}{2D Physics Based Game}
  (Uses doxygen)
\item
  \href{https://gitlab.cas.mcmaster.ca/smiths/cas741/tree/master/BlankProjectTemplate/Doc/TestReport}{Follow given template}
\item Examples are not perfect
\item Examples are intended to give you ideas, not to be strictly followed
\item You can modify/extend the test report template as appropriate
\ei

\end{frame}

%%%%%%%%%%%%%%%%%%%%%%%%%%%%%%%%%%%%%%%%%%%%%%%%%%%%%%

\begin{frame}
\frametitle{Questions?}
\begin{itemize}
\item Questions Final documentation?
\end{itemize}
\end{frame}

%%%%%%%%%%%%%%%%%%%%%%%%%%%%%%%%%%%%%%%%%%%%%%%%%%%%%%

\begin{frame}
\frametitle{Discussion}

\bi
\item Thoughts on documentation
\bi
\item SRS
\item VnV Plan
\item MG
\item MIS
\ei
\item Thoughts on technology
\bi
\item Git
\item GitHub
\item LaTeX
\item Make
\item Your programming language
\ei
\ei

\end{frame}

%%%%%%%%%%%%%%%%%%%%%%%%%%%%%%%%%%%%%%%%%%%%%%%%%%%%%%

\begin{frame}
\frametitle{Discussion: Course Content}

\bi
\item What ideas from the course will you continue to use?
\item Thoughts on
\bi
\item Drasil
\item Assurance cases
\ei
\item Other thoughts?
\ei

\end{frame}

%%%%%%%%%%%%%%%%%%%%%%%%%%%%%%%%%%%%%%%%%%%%%%%%%%%%%%

\begin{frame}
\frametitle{Discussion: Course Structure}

\bi
\item What can be done to improve the course on its next iteration?
\item Increase number of reviewers for GitHub issue creation?
\item How to get more discussion in class?
\ei

\end{frame}

%%%%%%%%%%%%%%%%%%%%%%%%%%%%%%%%%%%%%%%%%%%%%%%%%%%%%%

\end{document}