%\documentclass[t, 12pt, numbers, fleqn, handout]{beamer}
\documentclass[t,12pt,numbers,fleqn]{beamer}
%\documentclass[ignorenonframetext]{beamer}

\newif\ifquestions
%\questionstrue
\questionsfalse

\usepackage{pgfpages} 
\usepackage{hyperref}
\hypersetup{colorlinks=true,
    linkcolor=blue,
    citecolor=blue,
    filecolor=blue,
    urlcolor=blue,
    unicode=false}
\urlstyle{same}

\usepackage{booktabs}
\usepackage{multirow}

\bibliographystyle{plain}

%\usetheme{Iimenau}

\useoutertheme{split} %so the footline can be seen, without needing pgfpages

%\pgfpagesuselayout{resize to}[letterpaper,border shrink=5mm,landscape]  %if this is uncommented, the hyperref links do not work

\mode<presentation>{}

\input{../def-beamer}

\newcommand{\topic}{14 Module Interface Specification (MIS)}

\input{../titlepage}

\begin{document}

\input{../footline}

%%%%%%%%%%%%%%%%%%%%%%%%%%%%%%%%%%%%%%%%%%%%%%%%%%%%%%

\begin{frame}
\frametitle{Module Interface Specification (MIS)}

\bi
\item Administrative details
\item Questions?
\item Module guide example
\item Integration testing
\item Mathematical review (\cite{HoffmanAndStrooper1995} and separate slides)
\begin{itemize}
\item Multiple assignment statement
\item Conditional rules
\item etc.
\end{itemize}
\item MIS overview
\item Modules with external interaction
\item Abstract objects
\item Abstract data types
\item Generic MIS
\item Inheritance
\ei
\end{frame}

%%%%%%%%%%%%%%%%%%%%%%%%%%%%%%%%%%%%%%%%%%%%%%%%%%%%%%

\begin{frame}
\frametitle{Administrative Details}

\bi
\item GitHub issues for colleagues
\bi
\item Assigned 1 colleague (see \texttt{Repos.xlsx} in repo)
\item Provide at least 5 issues on their VnV plan
\item Grading as before
\item Due by Tuesday, Oct 31, 11:59 pm
\ei

\item For MG presentation, we'll try to use my laptop only
\item Template for MG in repo

\ei

\end{frame}

%%%%%%%%%%%%%%%%%%%%%%%%%%%%%%%%%%%%%%%%%%%%%%%%%%%%%%

\begin{frame}
\frametitle{Administrative Details: Deadlines}
~\newline
\begin{tabular}{l l l}
\textbf{MG Present} & Week 08 & Week of Oct 30\\
\textbf{MG} & Week 09 & Nov 8\\
MIS Present & Week 10 & Week of Nov 13\\
MIS & Week 11 & Nov 22\\
Impl.\ Present & Week 12 & Week of Nov 27\\
Final Documentation & Week 13 & Dec 6\\
\end {tabular}

\end{frame}

%%%%%%%%%%%%%%%%%%%%%%%%%%%%%%%%%%%%%%%%%%%%%%%%%%%%%%

\begin{frame}
\frametitle{Administrative Details: Presentation Schedule}

\bi
\item MG Present
\bi
\item \textbf{Tuesday: Xiaoye, Yuzhi, Devi, Keshav, Alex P, Paul}
\item \textbf{Friday: Shusheng, Jason, Geneva, Alex S, Isobel, Steven}
\ei
\item MIS Present
\bi
\item Tuesday: Isobel, Keshav, Paul
\item Friday: Shusheng, Xiaoye, Devi
\ei
\item Impl.\ Present
\bi
\item Tuesday: Alexander S., Steven, Alexandre P.
\item Friday: Jason, Geneva, Yuzhi
\ei

\ei

\end{frame}

%%%%%%%%%%%%%%%%%%%%%%%%%%%%%%%%%%%%%%%%%%%%%%%%%%%%%%

\begin{frame}
\frametitle{Questions?}
\begin{itemize}
\item Questions about Module Guide and the presentation?
\end{itemize}
\end{frame}

%%%%%%%%%%%%%%%%%%%%%%%%%%%%%%%%%%%%%%%%%%%%%%%%%%%%%%

\begin{frame}
\frametitle{Solar Water Heating System Example}

\bi
\item \href{https://github.com/smiths/swhs}{https://github.com/smiths/swhs}
\item Solve ODEs for temperature of water and PCM
\item Solve for energy in water and PCM
\item Generate plots
\ei

\begin{figure}
{
\includegraphics[width=0.5\textwidth]{../Figures/Tank.pdf}
}
\end{figure}

\end{frame}

%%%%%%%%%%%%%%%%%%%%%%%%%%%%%%%%%%%%%%%%%%%%%%%%%%%%%%

\begin{frame}
\frametitle{Anticipated Changes?}

\structure{What are some anticipated changes?}\\
~\newline
Hint: the software follows the Input $\rightarrow$ Calculate $\rightarrow$
Output design pattern

\end{frame}

%%%%%%%%%%%%%%%%%%%%%%%%%%%%%%%%%%%%%%%%%%%%%%%%%%%%%%

\begin{frame}
\frametitle{Anticipated Changes}

\bi
\item The specific hardware on which the software is to run
\item The format of the initial input data
\item The format of the input parameters
\item The constraints on the input parameters
\item The format of the output data
\item The constraints on the output results
\item How the governing ODEs are defined using the input parameters
\item How the energy equations are defined using the input parameters
\item How the overall control of the calculations is orchestrated
\item The implementation of the sequence data structure
\item The algorithm used for the ODE solver
\item The implementation of plotting data
\ei

\end{frame}

%%%%%%%%%%%%%%%%%%%%%%%%%%%%%%%%%%%%%%%%%%%%%%%%%%%%%%

\begin{frame}
\frametitle{Module Hierarchy by Secrets}
\begin{table}[h!]
\centering
\begin{tabular}{p{0.3\textwidth} p{0.6\textwidth}}
\toprule
\textbf{Level 1} & \textbf{Level 2}\\
\midrule

{Hardware-Hiding Module} & ~ \\
\midrule

\multirow{6}{0.3\textwidth}{Behaviour-Hiding Module} & Input Parameters Module\\
& Output Format Module\\
& Temperature ODEs Module\\
& Energy Equations Module\\ 
& Control Module\\
& Specification Parameters\\
\midrule

\multirow{3}{0.3\textwidth}{Software Decision Module} & {Sequence Data Structure Module}\\
& ODE Solver Module\\
& Plotting Module\\
\bottomrule

\end{tabular}
\caption{Module Hierarchy}
\label{TblMH}
\end{table}
\end{frame}

%%%%%%%%%%%%%%%%%%%%%%%%%%%%%%%%%%%%%%%%%%%%%%%%%%%%%%

\begin{frame}
\frametitle{Example Modules from SWHS}

\textbf{Hardware Hiding Modules}

\begin{description}
\item[Secrets:] The data structure and algorithm used to implement the virtual
  hardware.
\item[Services:] Serves as a virtual hardware used by the rest of the
  system. This module provides the interface between the hardware and the
  software. So, the system can use it to display outputs or to accept inputs.
\item[Implemented By:] OS
\end{description}

\end{frame}

%%%%%%%%%%%%%%%%%%%%%%%%%%%%%%%%%%%%%%%%%%%%%%%%%%%%%%

\begin{frame}
\frametitle{Example Modules from SWHS}

\textbf{Input Parameters Module}

\begin{description}
\item[Secrets:] The data structure for input parameters, how the
values are input and how the values are verified.  The load and verify secrets
are isolated to their own access programs (like submodules). %    This, combined
% with the fact that all of the services are invoked together, suggests that the
% one module one secret rule can be relaxed here.
\item[Services:] Gets input from user (including material properties, processing
  conditions, and numerical parameters), stores input and verifies that the
  input parameters comply with physical and software constraints. Throws an
  error if a parameter violates a physical constraint. Throws a warning if a
  parameter violates a software constraint.  % Stored parameters can be read
  % individually, but write access is only to redefine the entire set of inputs.
\item[Implemented By:] SWHS
\end{description}

\end{frame}

%%%%%%%%%%%%%%%%%%%%%%%%%%%%%%%%%%%%%%%%%%%%%%%%%%%%%%

\begin{frame}
\frametitle{Example Modules from SWHS}

\textbf{ODE Solver Module}

\begin{description}
\item[Secrets:] The algorithm to solve a system of first order ODEs initial
  value problem from a given starting time until the given event function shows
  termination.
\item[Services:] Solves an ODE using the governing equation, initial
  conditions, event function and numerical parameters.
\item[Implemented By:] Matlab
\end{description}

\end{frame}

%%%%%%%%%%%%%%%%%%%%%%%%%%%%%%%%%%%%%%%%%%%%%%%%%%%%%%

\begin{frame}[plain, fragile]

\frametitle{SWHS Uses Hierarchy (approximately)}

\begin{center}
\includegraphics[scale=0.55]{../Figures/UsesHierarchy.pdf}
\end{center}

\end{frame}

%%%%%%%%%%%%%%%%%%%%%%%%%%%%%%%%%%%%%%%%%%%%%%%%%%%%%%

\begin{frame}
\frametitle{Mesh Generator Example}
\vspace{-1cm}
\begin{figure}[H]
\includegraphics[scale=0.4, angle=0]{../Figures/ExampleMesh.pdf}
\end{figure}\end{frame}

%%%%%%%%%%%%%%%%%%%%%%%%%%%%%%%%%%%%%%%%%%%%%%%%%%%%%%

\begin{frame}
\frametitle{Mesh Generator Complex Circular Geometry}
\begin{figure}[H]
\includegraphics[scale=0.25]{../Figures/magnetron1.pdf}
\end{figure}\end{frame}

%%%%%%%%%%%%%%%%%%%%%%%%%%%%%%%%%%%%%%%%%%%%%%%%%%%%%%

\begin{frame}
\frametitle{Mesh Generator Example: Design Goals}
\begin{itemize}
\item Independent and flexible representation for each mesh entity
\item Complete separation of geometric data from the topology
\item The mesh generator should work with different coordinate systems
\item A flexible data structure to store sets of vertices, edges and triangles
\item Different mesh generation algorithms with a minimal amount of local changes
\end{itemize}
\end{frame}

%%%%%%%%%%%%%%%%%%%%%%%%%%%%%%%%%%%%%%%%%%%%%%%%%%%%%%

\begin{frame}
\frametitle{Example Mesh Gen Modular Decomposition}

%\vspace{-1.5cm}
% \begin{center}
% \includegraphics[width=0.5\textwidth]{DecompBySecretHierarchyExample.png}
% \end{center}

\href{https://gitlab.cas.mcmaster.ca/smiths/se2aa4_cs2me3/blob/master/Lectures/L14_ModuleDecompositionContinued/DecompBySecretHierarchyExample.png}{Link}

\end{frame}

%%%%%%%%%%%%%%%%%%%%%%%%%%%%%%%%%%%%%%%%%%%%%%%%%%%%%%

\begin{frame}
\frametitle{Another Mesh Generator Uses Hierarchy \cite{ElSheikhEtAl2004}}
\begin{figure}[H]
\includegraphics[scale=0.345]{../Figures/hierarchy_diagram.pdf}
\end{figure}
\end{frame}

%%%%%%%%%%%%%%%%%%%%%%%%%%%%%%%%%%%%%%%%%%%%%%%%%%%%%%

\begin{frame}
\frametitle{Module Testing}

\structure{Is it possible to begin testing before all of the modules have been
  implemented when there is a use relation between modules?}

\end{frame}

%%%%%%%%%%%%%%%%%%%%%%%%%%%%%%%%%%%%%%%%%%%%%%%%%%%%%%

\begin{frame}
\frametitle{Module Testing \cite{GhezziEtAl2003}}

\begin{itemize}

\item Scaffolding needed to create the environment in which the module should be
  tested
\item Stubs - a module used by the module under test
\item Driver - module activating the module under test
\end{itemize}

\end{frame}

%%%%%%%%%%%%%%%%%%%%%%%%%%%%%%%%%%%%%%%%%%%%%%%%%%%%%%

\begin{frame}
\frametitle{Testing a Functional Module \cite{GhezziEtAl2003}}

\includegraphics[scale=0.5]{../Figures/TestingAFunctModule.png}

\end{frame}

%%%%%%%%%%%%%%%%%%%%%%%%%%%%%%%%%%%%%%%%%%%%%%%%%%%%%%

\begin{frame}
\frametitle{Integration Testing}

\begin{itemize}
\item Big-bang approach
\begin{itemize}
\item First test individual modules in isolation
\item Then test integrated system
\end{itemize}
\item Incremental approach
\begin{itemize}
\item Modules are progressively integrated and tested
\item Can proceed both top-down and bottom-up according to the USES relation
\end{itemize}
\end{itemize}
\end{frame}

%%%%%%%%%%%%%%%%%%%%%%%%%%%%%%%%%%%%%%%%%%%%%%%%%%%%%%

\begin{frame}
\frametitle{Integration Testing and USES relation}

\begin{itemize}
\item If integration and test proceed bottom-up only need drivers
\item Otherwise if we proceed top-down only stubs are needed
\end{itemize}
\end{frame}

%%%%%%%%%%%%%%%%%%%%%%%%%%%%%%%%%%%%%%%%%%%%%%%%%%%%%%

\begin{frame}
\frametitle{Example \cite{GhezziEtAl2003}}

\begin{center}
\includegraphics[scale=0.35]{../Figures/Example.png}
\end{center}

\begin{itemize}
\item $M_1$ USES $M_2$ and $M_2$ IS\_COMPOSED\_OF \{$M_{2,1}$, $M_{2,2}$\}
\item \structure{In what order would you test these modules?}
\end{itemize}
\end{frame}

%%%%%%%%%%%%%%%%%%%%%%%%%%%%%%%%%%%%%%%%%%%%%%%%%%%%%%

\begin{frame}
\frametitle{Example \cite{GhezziEtAl2003}}

\begin{center}
\includegraphics[scale=0.35]{../Figures/Example.png}
\end{center}

\begin{itemize}
\item $M_1$ USES $M_2$ and $M_2$ IS\_COMPOSED\_OF \{$M_{2,1}$, $M_{2,2}$\}
\item Case 1
\begin{itemize}
\item Test $M_1$ providing a stub for $M_2$ and a driver for $M_1$
\item Then provide an implementation for $M_{2,1}$ and a stub for $M_{2,2}$
\end{itemize}
\item Case 2
\begin{itemize}
\item Implement $M_{2,2}$ and test it by using a driver
\item Implement $M_{2,1}$ and test the combination of $M_{2,1}$ and $M_{2,2}$
  (i.e.\ $M_2$) by using a driver
\item Finally implement $M_1$ and test it with $M_2$ using a driver for $M_1$
\end{itemize}
\end{itemize}
\end{frame}

%%%%%%%%%%%%%%%%%%%%%%%%%%%%%%%%%%%%%%%%%%%%%%%%%%%%%%

\begin{frame}
\frametitle{Overview of MIS}
\begin{itemize}
\item See Hoffman and Strooper \cite{HoffmanAndStrooper1995}
\item The MIS precisely specifies the modules observable behaviour - what the module does
\item The MIS does not specify the internal design
\item The idea of an MIS is inspired by the principles of software engineering
\item Advantages
\begin{itemize}
\item Improves many software qualities
\item Programmers can work in parallel
\item Assumptions about how the code will be used are recorded
\item Test cases can be decided on early, and they benefit from a clear specification of the behaviour
\item A well designed and documented MIS is easier to read and understand than complex code
\item Can use the interface without understanding details
\end{itemize}
\end{itemize}
\end{frame}

%%%%%%%%%%%%%%%%%%%%%%%%%%%%%%%%%%%%%%%%%%%%%%%%%%%%%%

\begin{frame}
\frametitle{Overview of MIS}
\begin{itemize}
\item Options for specifying an MIS
\begin{itemize}
\item Trace specification
\item Pre and post conditions specification
\item Input/output specification
\item \structure{Before/after specification - module state machine}
\item Algebraic specification
\end{itemize}
\item Best to follow a template
\end{itemize}
\end{frame}

%%%%%%%%%%%%%%%%%%%%%%%%%%%%%%%%%%%%%%%%%%%%%%%%%%%%%%

\begin{frame}
\frametitle{MIS Template}

\begin{itemize}
\item Uses
\begin{itemize}
\item Imported constants, data types and access programs
\end{itemize}
\item Syntax
\begin{itemize}
\item Exported constants and types
\item Exported functions (access routine interface syntax)
\end{itemize}
\item Semantics
\begin{itemize}
\item State variables
\item State invariants
\item Assumptions
\item Access routine semantics
\item Local functions
\item Local types
\item Local constants
\item Considerations
\end{itemize}
\end{itemize}
\end{frame}

%%%%%%%%%%%%%%%%%%%%%%%%%%%%%%%%%%%%%%%%%%%%%%%%%%%%%%

\begin{frame}
\frametitle{MIS Uses Section}
\begin{itemize}
\item Specify imported constants
\item Specify imported types
\item Specify imported access programs
\item The specification of one module will often depend on using the interface specified by another module
\item When there are many modules the uses information is very useful for navigation of the documentation
\item Documents the use relation between modules
\end{itemize}
\end{frame}

%%%%%%%%%%%%%%%%%%%%%%%%%%%%%%%%%%%%%%%%%%%%%%%%%%%%%%

\begin{frame}
\frametitle{MIS Syntax Section}
\begin{itemize}
\item Specify exported constants
\item Specify exported types
\item Specify access routine names, the input and output parameter types and exceptions
\item Show access routines in tabular form
\begin{itemize}
\item Important design decisions are made at this point
\item The goal is to have the syntax match many implementation languages
\end{itemize}
\end{itemize}
\end{frame}

%%%%%%%%%%%%%%%%%%%%%%%%%%%%%%%%%%%%%%%%%%%%%%%%%%%%%%

\begin{frame}
\frametitle{Syntax of a Sequence Module}

\textbf{Exported Constants}\\
~\newline
MAX\_SIZE = 100\\

\end{frame}

%%%%%%%%%%%%%%%%%%%%%%%%%%%%%%%%%%%%%%%%%%%%%%%%%%%%%%

\begin{frame}
\frametitle{Syntax of a Sequence Module Continued}

\textbf{Exported Access Programs}\\
~\newline
\begin{tabular}{| l | l | l | l |}
\hline
\textbf{Routine name} & \textbf{In} & \textbf{Out} & \textbf{Exceptions}\\
\hline
seq\_init & ~ & ~ & ~\\
\hline
seq\_add & integer, integer & ~ & FULL, POS\\
\hline
seq\_del & integer & ~ & POS\\
\hline
seq\_setval & integer, integer & ~ & POS\\
\hline
seq\_getval & integer & integer & POS\\
\hline
seq\_size & ~ & integer & ~\\
\hline

\end{tabular}

\end{frame}

%%%%%%%%%%%%%%%%%%%%%%%%%%%%%%%%%%%%%%%%%%%%%%%%%%%%%%

\begin{frame}
\frametitle{MIS Semantics Section}

\begin{itemize}
\item State variables
\begin{itemize}
\item Give state variable(s) name and type
\item State variables define the state space
\item If a module has state then it will have ``memory''
\end{itemize}
\item State invariant
\begin{itemize}
\item A predicate on the state space that restricts the ``legal'' states of the module
\item After every access routine call, the state should satisfy the invariant
\item Cannot have a state invariant without state variables
\item Just stating the invariant does not ``enforce'' it, the access routine semantics need to maintain it
\item Useful for understandabilty, testing and for proof
\end{itemize}
\end{itemize}

\end{frame}

%%%%%%%%%%%%%%%%%%%%%%%%%%%%%%%%%%%%%%%%%%%%%%%%%%%%%%

\begin{frame}
\frametitle{Semantics Section Continued}

\begin{itemize}
\item Local functions, local types and local constants
\begin{itemize}
\item Declared for specification purposes only
\item Not available at run time
\item Helpful to make complex specifications easier to read
\end{itemize}
\item Considerations
\begin{itemize}
\item For information that does not fit elsewhere
\item Useful to tell the user if the module violates a quality criteria
\end{itemize}
\end{itemize}

\end{frame}

%%%%%%%%%%%%%%%%%%%%%%%%%%%%%%%%%%%%%%%%%%%%%%%%%%%%%%

\begin{frame}
\frametitle{Sequence MIS Semantics}

\textbf{State Variables}\\
$s$: sequence of integer
~\newline

\textbf{State Invariant}\\
$| s | \leq \mathrm{MAX\_SIZE}$
~\newline

\textbf{Assumptions}\\
seq\_init() is called before any other access program

\end{frame}

%%%%%%%%%%%%%%%%%%%%%%%%%%%%%%%%%%%%%%%%%%%%%%%%%%%%%%

\begin{frame}
\frametitle{Sequence MIS Semantics Continued}

\textbf{Access Routine Semantics}\\
~\newline
seq\_init():
\begin{itemize}
\item transition: $s := < >$
\item exception: none
\end{itemize}
~\newline
\noindent seq\_add($i, p$):
\begin{itemize}
\item transition: $s := s[0..i-1] || <p> || s[i..|s|-1]$
\item exception: $exc := (|s| = \mathrm{MAX\_SIZE} \Rightarrow  \mathrm{FULL} ~ | ~ i \notin [0..|s|] \Rightarrow
\mathrm{POS})$
\end{itemize}

\end{frame}

%%%%%%%%%%%%%%%%%%%%%%%%%%%%%%%%%%%%%%%%%%%%%%%%%%%%%%

\begin{frame}
\frametitle{Access Routine Semantics Continued}

\noindent seq\_del($i$):
\begin{itemize}
\item transition: $s := s[0..i-1] || s[i+1..|s|-1]$
\item exception:  $exc := (i \notin [0..|s|-1] \Rightarrow \mathrm{POS})$
\end{itemize}
~\newline
\noindent seq\_setval($i, p$):
\begin{itemize}
\item transition: $s[i] := p$
\item exception: $exc := (i \notin [0..|s|-1] \Rightarrow \mathrm{POS})$
\end{itemize}
~\newline
\noindent seq\_getval($i$):
\begin{itemize}
\item output: $out := s[i]$
\item exception: $exc := (i \notin [0..|s|-1] \Rightarrow \mathrm{POS})$
\end{itemize}

\end{frame}

%%%%%%%%%%%%%%%%%%%%%%%%%%%%%%%%%%%%%%%%%%%%%%%%%%%%%%

\begin{frame}
\frametitle{Access Routine Semantics Continued}

\noindent seq\_size():
\begin{itemize}
\item output: $out := | s |$
\item exception: none
\end{itemize}

\end{frame}

%%%%%%%%%%%%%%%%%%%%%%%%%%%%%%%%%%%%%%%%%%%%%%%%%%%%%%

\begin{frame}

\frametitle{Exception Signaling}

\begin{itemize}
\item Useful to think about exceptions in the design process
\item Will need to decide how exception signalling will be done
\begin{itemize}
\item A special return value, a special status parameter, a global variable
\item Invoking an exception procedure
\item Using built-in language constructs
\end{itemize}
\item Caused by errors made by programmers, not by users
\item Write code so that it avoid exceptions
\item Exceptions will be particularly useful during testing
\end{itemize}

\end{frame}

%%%%%%%%%%%%%%%%%%%%%%%%%%%%%%%%%%%%%%%%%%%%%%%%%%%%%%

\begin{frame}

\frametitle{Assumptions versus Exceptions}

\begin{itemize}
\item The assumptions section lists assumptions the module developer is permitted to make about the programmer's behaviour
\item Assumptions are expressed in prose
\item Use assumptions to simplify the MIS and to reduce the complexity of the final implementation
\item Interface design should provide the programmer with a means to check so that they can avoid exceptions
\item When an exceptions occurs no state transitions should take place, any output is {\it don't care}
\end{itemize}

\end{frame}

%%%%%%%%%%%%%%%%%%%%%%%%%%%%%%%%%%%%%%%%%%%%%%%%%%%%%%

\begin{frame}

\frametitle{Quality Criteria}

\begin{itemize}
\item Consistent
\begin{itemize}
\item Name conventions
\item Ordering of parameters in argument lists
\item Exception handling, etc.
\end{itemize}
\item Essential - omit unnecessary features
\item General - cannot always predict how the module will be used
\item As implementation independent as possible
\item Minimal - avoid access routines with two potentially independent services
\item High cohesion - components are closely related
\item Low coupling - not strongly dependent on other modules
\item Opaque - information hiding
\end{itemize}

\end{frame}

%%%%%%%%%%%%%%%%%%%%%%%%%%%%%%%%%%%%%%%%%%%%%%%%%%%%%%

\begin{frame}

\frametitle{SWHS Example}

Look at SWHS repo

\end{frame}

%%%%%%%%%%%%%%%%%%%%%%%%%%%%%%%%%%%%%%%%%%%%%%%%%%%%%%

\begin{frame}
\frametitle{Modules with External Interaction}

\begin{itemize}
\item In general, some modules may interact with the environment or other modules
%\item The Display Point Masses Applet module does this
\item Environment might include the keyboard, the screen, the file system, motors, sensors, etc.
\item Sometimes the interaction is informally specified using prose (natural language)
\item Can introduce an environment variable
\begin{itemize}
\item Name, type
\item Interpretation
\end{itemize}
\item Environment variables include the screen, the state of a motor (on, direction of rotation, power level, etc.),
the position of a robot

\end{itemize}

\end{frame}

%%%%%%%%%%%%%%%%%%%%%%%%%%%%%%%%%%%%%%%%%%%%%%%%%%%%%%

\begin{frame}
\frametitle{External Interaction Continued}

\begin{itemize}
\item Some external interactions are hidden
\begin{itemize}
\item Present in the implementation, but not in the MIS
\item An example might be OS memory allocation calls
\end{itemize}
\item External interaction described in the MIS
\begin{itemize}
\item Naming access programs of the other modules
\item Specifying how the other module's state variables are changed
\item The MIS should identify what external modules are used
\end{itemize}
\end{itemize}

\end{frame}

%%%%%%%%%%%%%%%%%%%%%%%%%%%%%%%%%%%%%%%%%%%%%%%%%%%%%%

\begin{frame}
\frametitle{MIS for GUI Modules}

\begin{itemize}
\item Could introduce an environment variable
\item window: sequence [RES\_H][RES\_V] of pixelT
\begin{itemize}
\item Where window[r][c] is the pixel located at row r and column c, with numbering zero-relative and beginning at the
upper left corner
\item Would still need to define pixelT
\end{itemize}
\item Could formally specify the environment variable transitions
\item More often it is reasonable to specify the transition in prose
\item In some cases the proposed GUI might be shown by rough sketches
\end{itemize}

\end{frame}

%%%%%%%%%%%%%%%%%%%%%%%%%%%%%%%%%%%%%%%%%%%%%%%%%%%%%%

\begin{frame}
\frametitle{Display Point Masses Module Syntax}

\textbf{Exported Access Programs}\\
~\newline
\begin{tabular}{| l | l | l | l |}
\hline
\textbf{Routine name} & \textbf{In} & \textbf{Out} & \textbf{Exc}\\
\hline
DisplayPointMassesApplet & ~ & DisplayPointMassesApplet & ~\\
\hline
paint & ~ & ~ & ~\\
\hline
\end{tabular}

\end{frame}

%%%%%%%%%%%%%%%%%%%%%%%%%%%%%%%%%%%%%%%%%%%%%%%%%%%%%%

\begin{frame}
\frametitle{Display Point Masses Module Semantics}

\textbf{Environment Variables}\\
$\mathit{win}:$ 2D sequence of pixels displayed within a web-browser \\

\noindent DisplayPointMassesApplet():
\begin{itemize}
\item transition: The state of the abstract object ListPointMasses is modified as follows:\\
ListPointMasses.init()\\
ListPointMasses.add(0, PointMassT(20, 20, 10))\\
ListPointMasses.add(1, PointMassT(120, 200, 20))\\
...
\end{itemize}

\noindent paint():
\begin{itemize}
\item transition $win := $ Modify window so that the point masses in ListPointMasses 
are plotted as circles.  The centre of each circles should be the corresponding x and y coordinates and the radius should
be the mass of the point mass.
\end{itemize}

\end{frame}

%%%%%%%%%%%%%%%%%%%%%%%%%%%%%%%%%%%%%%%%%%%%%%%%%%%%%%

\begin{frame}
\frametitle{Specification of ADTs}
\begin{itemize}
\item Similar template to abstract objects
\item ``Template Module'' as opposed to ``Module''
\item ``Exported Types'' that are abstract use a ?
\begin{itemize}
\item pointT = ?
\item pointMassT = ?
\end{itemize}
\item Access routines know which abstract object called them
\item Use ``self'' to refer to the current abstract object
\item Use a dot ``.'' to reference methods of an abstract object
\begin{itemize}
\item $p.\mbox{xcoord()}$
\item $\mathit{self}.pt.\mbox{dist}(p.\mbox{point}())$
\end{itemize}
\item Similar notation to Java
\item The syntax of the interface in C is different
\end{itemize}
\end{frame}

%%%%%%%%%%%%%%%%%%%%%%%%%%%%%%%%%%%%%%%%%%%%%%%%%%%%%%

\begin{frame}
\frametitle{Syntax Line ADT Module}

\textbf{Template Module}\\
~\newline
lineADT\\
~\newline
\textbf{Uses}\\
~\newline
pointADT\\
~\newline
\textbf{Exported Types}\\
~\newline
lineT = ?

\end{frame}

%%%%%%%%%%%%%%%%%%%%%%%%%%%%%%%%%%%%%%%%%%%%%%%%%%%%%%

\begin{frame}
\frametitle{Syntax Line ADT Module Continued}

\begin{tabular}{| l | l | l | l |}
\hline
\textbf{Routine name} & \textbf{In} & \textbf{Out} & \textbf{Exceptions}\\
\hline
new lineT & pointT, pointT & lineT & ~\\
\hline
start & ~ & pointT & ~\\
\hline
end & ~ & pointT & ~\\
\hline 
length & ~ & real & ~\\
\hline
midpoint & ~ & pointT & ~\\
\hline
rotate & real & ~ & ~\\
\hline
\end{tabular}

\end{frame}

%%%%%%%%%%%%%%%%%%%%%%%%%%%%%%%%%%%%%%%%%%%%%%%%%%%%%%

\begin{frame}
\frametitle{Semantics Line ADT Module}

\textbf{State Variables}\\
~\newline
$s$: pointT\\
$e$: pointT\\
~\newline
\textbf {State Invariant}\\
~\newline
None\\
~\newline
\textbf{Assumptions}\\
~\newline
None

\end{frame}

%%%%%%%%%%%%%%%%%%%%%%%%%%%%%%%%%%%%%%%%%%%%%%%%%%%%%%

\begin{frame}
\frametitle{Access Routine Semantics Line ADT Module}

\noindent new lineT ($p_1, p_2$):
\begin{itemize}
\item transition: $s, e := p_1, p_2$
\item output: $out := \mathit{self}$
\item exception: none
\end{itemize}

\noindent start:
\begin{itemize}
\item output: $out := s$
\item exception: none
\end{itemize}

\noindent end:
\begin{itemize}
\item output: $out := e$
\item exception: none
\end{itemize}

\end{frame}

%%%%%%%%%%%%%%%%%%%%%%%%%%%%%%%%%%%%%%%%%%%%%%%%%%%%%%

\begin{frame}
\frametitle{Access Routine Semantics Continued}

\noindent length:
\begin{itemize}
\item output: $out := s.\mathrm{dist}(e)$
\item exception: none
\end{itemize}

\noindent midpoint:
\begin{itemize}
\item output: $out := $
$$\mathrm{new~} \mathrm{pointT} (\mathrm{avg}(s.\mathrm{xcoord}, e.\mathrm{xcoord}), \mathrm{avg}(s.\mathrm{ycoord},
e.\mathrm{ycoord}))$$
\item exception: none
\end{itemize}

\noindent rotate ($\phi$):\\
$\phi$ is in radians
\begin{itemize}
\item transition: $s.\mathrm{rotate}(\phi), e.\mathrm{rotate}(\phi)$
\item exception: none
\end{itemize}

\end{frame}

%%%%%%%%%%%%%%%%%%%%%%%%%%%%%%%%%%%%%%%%%%%%%%%%%%%%%%

\begin{frame}
\frametitle{Line ADT Local Functions}

\textbf{Local Functions}\\
~\newline
avg: real $\times$ real $\rightarrow$ real

\noindent avg($x_1, x_2$) $\equiv \frac{x_1 + x_2}{2}$

\end{frame}

%%%%%%%%%%%%%%%%%%%%%%%%%%%%%%%%%%%%%%%%%%%%%%%%%%%%%%

% \begin{frame}
% \frametitle{ADTs in C}

% \begin{itemize}
% \item Each object will be a pointer to a struct that holds its state
% \item The interface for each routine, other than initialization routines, will need this pointer
% \item There is an implicit object of the ADT type as the first input parameter in all routines other than init
% \item Need to add a prefix to all routines, other than initialization routines, to avoid naming conflicts
% \item For PointADT, using prefix {\tt pt\_}
% \begin{itemize}
% \item {\tt typedef struct Point* pointT;}
% \item \uncover<2->{\tt pointT pt\_init(float x, float y);}
% \item \uncover<3->{\tt float pt\_xcoord(pointT self);}
% \item \uncover<4->{\tt float pt\_ycoord(pointT self);}
% \item \uncover<5->{\tt float pt\_dist(pointT self, pointT p);}
% \end{itemize}
% \item For PointMassADT use prefix {\tt pm\_}
% \end{itemize}
% \end{frame}

%%%%%%%%%%%%%%%%%%%%%%%%%%%%%%%%%%%%%%%%%%%%%%%%%%%%%%

\begin{frame}
\frametitle{Generic Modules}

\begin{itemize}
\item What if we have a sequence of integers, instead of a sequence of point masses?
\item What if we want a stack of integers, or characters, or pointT, or pointMassT?
\item Do we need a new specification for each new abstract object?
\item No, we can have a single abstract specification implementing a family of abstract objects that are distinguished
only by a few variabilities
\item Rather than duplicate nearly identical modules, we parameterize one \structure{generic module} with respect to
type(s)
\item Advantages
\begin{itemize}
\item Eliminate chance of inconsistencies between modules
\item Localize effects of possible modifications
\item Reuse
\end{itemize}
\end{itemize}

\end{frame}

%%%%%%%%%%%%%%%%%%%%%%%%%%%%%%%%%%%%%%%%%%%%%%%%%%%%%%

\begin{frame}
\frametitle{Generic Stack Module Syntax}

\textbf{Generic Module}\\
~\newline
Stack(T)\\
~\newline
\textbf{Exported Constants}\\
~\newline
$\mbox{MAX\_SIZE} = 100$
~\newline

\textbf{Exported Access Programs}\\
~\newline
\begin{tabular}{| l | l | l | l |}
\hline
\textbf{Routine name} & \textbf{In} & \textbf{Out} & \textbf{Exceptions}\\
\hline
... & ... & ... & ...\\
\hline
\end{tabular}

\end{frame}

%%%%%%%%%%%%%%%%%%%%%%%%%%%%%%%%%%%%%%%%%%%%%%%%%%%%%%

\begin{frame}
\frametitle{Stack Module Syntax}

\textbf{Exported Access Programs}\\
~\newline
\begin{tabular}{| l | l | l | l |}
\hline
\textbf{Routine name} & \textbf{In} & \textbf{Out} & \textbf{Exceptions}\\
\hline
s\_init & ~ & ~ & ~\\
\hline
s\_push & T & ~ & FULL\\
\hline
s\_pop & ~ & ~ & EMPTY\\
\hline
s\_top & ~ & T & EMPTY\\
\hline
s\_depth & ~ & integer & ~\\
\hline
\end{tabular}

\end{frame}

%%%%%%%%%%%%%%%%%%%%%%%%%%%%%%%%%%%%%%%%%%%%%%%%%%%%%%

\begin{frame}
\frametitle{Semantics}

\textbf{State Variables}\\
~\newline
\uncover<2->{\structure<2>{$s$: sequence of T}}\\
~\newline
\textbf{State Invariant}\\
~\newline
\uncover<3->{\structure<3>{$|s| \leq \mbox{MAX\_SIZE}$}}\\
~\newline
\textbf{Assumptions}\\
~\newline
\uncover<4->{\structure<4>{s\_init() is called before any other access routine}}\\

\end{frame}

%%%%%%%%%%%%%%%%%%%%%%%%%%%%%%%%%%%%%%%%%%%%%%%%%%%%%%

\begin{frame}
\frametitle{Access Routine Semantics}

s\_init():
\begin{itemize}
\item transition: \uncover<2->{\structure<2>{$s := <>$}}
\item exception: \uncover<3->{\structure<3>{none}}
\end{itemize}

\noindent s\_push(x):
\begin{itemize}
\item transition: \uncover<4->{\structure<4>{$s := s || <x>$}}
\item exception: \uncover<5->{\structure<5>{$exc := (|s| = \mbox{MAX\_SIZE} \Rightarrow \mbox{FULL})$}}
\end{itemize}

\noindent s\_pop():
\begin{itemize}
\item transition: \uncover<6->{\structure<6>{$s := s[0..|s| - 2]$}}
\item exception: \uncover<7->{\structure<7>{$exc := (|s| = 0 \Rightarrow \mbox{EMPTY})$}}
\end{itemize}

\end{frame}

%%%%%%%%%%%%%%%%%%%%%%%%%%%%%%%%%%%%%%%%%%%%%%%%%%%%%%

\begin{frame}
\frametitle{Access Routine Semantics Continued}

\noindent s\_top():
\begin{itemize}
\item output: \uncover<2->{\structure<2>{$out := s[|s| - 1]$}}
\item exception: \uncover<3->{\structure<3>{$exc := (|s| = 0 \Rightarrow \mbox{EMPTY})$}}
\end{itemize}

\noindent s\_depth():
\begin{itemize}
\item output: \uncover<4->{\structure<4>{$out := |s|$}}
\item exception: \uncover<5->{\structure<5>{none}}
\end{itemize}

\end{frame}

%%%%%%%%%%%%%%%%%%%%%%%%%%%%%%%%%%%%%%%%%%%%%%%%%%%%%%

\begin{frame}
\frametitle{Stack Module Properties}

$\{ \mathit{true} \}$

~~~~~s\_init()

$\{ | s' | = 0 \}$

~\newline
$\{ | s | < \mbox{MAX\_SIZE} \}$

~~~~~s\_push($x$)

$\{ | s' | = | s | + 1 \wedge s'[|s'| - 1] = x \wedge s'[0..|s|-1] = s[0..|s|-1] \}$

~\newline
$\{ |s| < \mbox{MAX\_SIZE} \}$

~~~~~s\_push($x$)\\
~~~~~s\_pop()

$s' = s$

\end{frame}

%%%%%%%%%%%%%%%%%%%%%%%%%%%%%%%%%%%%%%%%%%%%%%%%%%%%%%

\begin{frame}
\frametitle{Object Oriented Design}
\begin{itemize}
\item One kind of module, ADT, called class
\item A class exports operations (procedures) to manipulate instance objects (often called methods)
\item Instance objects accessible via references
\item Can have multiple instances of the class (class can be thought of as roughly corresponding to the notion of a type)
\end{itemize}
\end{frame}

%%%%%%%%%%%%%%%%%%%%%%%%%%%%%%%%%%%%%%%%%%%%%%%%%%%%%%

\begin{frame}
\frametitle{Inheritance}
\begin{itemize}
\item Another relation between modules (in addition to USES and IS\_COMPONENT\_OF)
\item ADTs may be organized in a hierarchy
\item Class B may specialize class A
\begin{itemize}
\item B inherits from A
\item Conversely, A generalizes B
\end{itemize}
\item A is a superclass of B
\item B is a subclass of A
\end{itemize}
\end{frame}

%%%%%%%%%%%%%%%%%%%%%%%%%%%%%%%%%%%%%%%%%%%%%%%%%%%%%%

\begin{frame}
\frametitle{Template Module Employee}

\begin{tabular}{| l | l | l | l |}
\hline
\textbf{Routine name} & \textbf{In} & \textbf{Out} & \textbf{Except}\\
\hline
Employee & string, string, moneyT & Employee & ~\\
\hline
first\_Name & ~ & string & ~\\
\hline
last\_Name & ~ & string & ~\\
\hline
where & ~ & siteT & ~\\
\hline
salary & ~ & moneyT & ~\\
\hline
fire & ~ & ~ & ~\\
\hline
assign & siteT & ~ & ~\\
\hline
\end{tabular}\\

\end{frame}

%%%%%%%%%%%%%%%%%%%%%%%%%%%%%%%%%%%%%%%%%%%%%%%%%%%%%%

\begin{frame}
\frametitle{Inheritance Examples}

\textbf{Template Module}
Administrative\_Staff \textbf{inherits} Employee\\
~\newline

\begin{tabular}{| l | l | l | l |}
\hline
\textbf{Routine name} & \textbf{In} & \textbf{Out} & \textbf{Exception}\\
\hline
do\_this & folderT & ~ & ~\\
\hline
\end{tabular}\\
~\newline

\textbf{Template Module}
Technical\_Staff \textbf{inherits} Employee\\
~\newline

\begin{tabular}{| l | l | l | l |}
\hline
\textbf{Routine name} & \textbf{In} & \textbf{Out} & \textbf{Exception}\\
\hline
get\_skill & ~ & skillT & ~\\
\hline
def\_skill & skillT & ~ & ~\\
\hline
\end{tabular}\\

\end{frame}

%%%%%%%%%%%%%%%%%%%%%%%%%%%%%%%%%%%%%%%%%%%%%%%%%%%%%%

\begin{frame}
\frametitle{Inheritance Continued}

\begin{itemize}
\item A way of building software incrementally
\item Useful for long lived applications because new features can be added without breaking the old applications
\item A subclass defines a subtype
\item A subtype is substitutable for the parent type
\item Polymorphism - a variable referring to type A can refer to an object of type B if B is a subclass of A
\item Dynamic binding - the method invoked through a reference depends on the type of the object associated with the
reference at runtime
\item All instances of the sub-class are instances of the super-class, so the type of the sub-class is a subtype
\item All instances of Administrative\_Staff and Technical\_Staff are instances of Employee
\end{itemize}
\end{frame}

%%%%%%%%%%%%%%%%%%%%%%%%%%%%%%%%%%%%%%%%%%%%%%%%%%%%%%

% \begin{frame}
% \frametitle{Inheritance Continued}

% emp1, emp2: Employee\\
% emp3: Technical\_Staff\\

% emp1 = Administrative\_Staff() \uncover<2->{\structure{$\surd$}}\\
% emp2 = Technical\_Staff() \uncover<3->{\structure{$\surd$}}\\
% emp3 = emp1 \uncover<4->{\structure{$\times$}}\\
% emp3 = (Technical\_Staff) emp1 \uncover<5->{\structure{$\surd$}}\\
% ~\newline
% \uncover<6->{Polymorphism: type of RHS must be a subtype of the LHS}
% \end{frame}

%%%%%%%%%%%%%%%%%%%%%%%%%%%%%%%%%%%%%%%%%%%%%%%%%%%%%%

\begin{frame}
\frametitle{Dynamic Binding}

\begin{itemize}
\item Many languages, like C, use static type checking
\item OO languages use dynamic type checking as the default
\item There is a difference between a \structure{type} and a \structure{class} once we know this
\begin{itemize}
\item Types are known at compile time
\item The class of an object may be known only at run time
\end{itemize}
\end{itemize}
\end{frame}

%%%%%%%%%%%%%%%%%%%%%%%%%%%%%%%%%%%%%%%%%%%%%%%%%%%%%%

\begin{frame}
\frametitle{Point ADT Module}

\textbf{Template Module}\\
~\newline
PointT\\
~\newline
\textbf{Uses}\\
~\newline
N/A\\
~\newline
\textbf{Syntax}\\
~\newline
\textbf{Exported Types}\\
~\newline
PointT = ?\\

\end{frame}

%%%%%%%%%%%%%%%%%%%%%%%%%%%%%%%%%%%%%%%%%%%%%%%%%%%%%%

\begin{frame}
\frametitle{Point ADT Module Continued}

\textbf{Exported Access Programs}\\
~\newline
\begin{tabular}{| l | l | l | l |}
\hline
\textbf{Routine name} & \textbf{In} & \textbf{Out} & \textbf{Exceptions}\\
\hline
new PointT & real, real & PointT & ~\\
\hline
xcoord & ~ & real & ~\\
\hline
ycoord & ~ & real & ~\\
\hline
dist & PointT & real & ~\\
\hline
\end{tabular}

~\newline
\textbf{Semantics}\\
~\newline
\textbf{State Variables}\\
~\newline
$xc$: real\\
$yc$: real

\end{frame}

%%%%%%%%%%%%%%%%%%%%%%%%%%%%%%%%%%%%%%%%%%%%%%%%%%%%%%

\begin{frame}
\frametitle{Point Mass ADT Module}

\textbf{Template Module}\\
~\newline
PointMassT \textbf{inherits} PointT\\
~\newline
\textbf{Uses}\\
~\newline
PointT\\
~\newline
\textbf{Syntax}\\
~\newline
\textbf{Exported Types}\\
~\newline
PointMassT = ?\\

\end{frame}

%%%%%%%%%%%%%%%%%%%%%%%%%%%%%%%%%%%%%%%%%%%%%%%%%%%%%%

\begin{frame}
\frametitle{Point Mass ADT Module Continued}

\textbf{Exported Access Programs}\\
~\newline
\begin{tabular}{| l | l | l | l |}
\hline
\textbf{Routine name} & \textbf{In} & \textbf{Out} & \textbf{Exceptions}\\
\hline
new PointMassT & real, real, real & PointMassT & NegMassExcep\\
\hline
mval & ~ & real & ~\\
\hline 
force & PointMassT & real & ~\\
\hline
fx & PointMassT & real & ~\\
\hline
\end{tabular}

~\newline
\textbf{Semantics}\\
~\newline
\textbf{State Variables}\\
~\newline
$ms$: real

\end{frame}

%%%%%%%%%%%%%%%%%%%%%%%%%%%%%%%%%%%%%%%%%%%%%%%%%%%%%%

\begin{frame}
\frametitle{Point Mass ADT Module Semantics}

\noindent new PointMassT($x, y, m$):
\begin{itemize}
\item transition: $xc, yc, ms := x, y, m$
\item output: $out := \mathit{self}$
\item exception: $exc := (m < 0 \Rightarrow \mbox{NegativeMassException})$
\end{itemize}
~\newline
\noindent force($p$):
\begin{itemize}
\item output: $$out := \mathrm{UNIVERAL\_G} \frac{\mathit{self}.ms \times p.ms}
{\mathit{self}.\mathrm{dist}(p)^2}$$
\item exception: none
\end{itemize}

\end{frame}

%%%%%%%%%%%%%%%%%%%%%%%%%%%%%%%%%%%%%%%%%%%%%%%%%%%%%%

\begin{frame}
\frametitle{Examples}

\begin{itemize}
\item Example Point Line and Circle
\item Example Robot Path
\item Example Vector Space
\item Example Othello Program
\item Example Maze Formal Specification (Dr.\ v.\ Mohrenschildt)
\item Mustafa ElSheikh Mesh Generator \cite{CaretteEtAl2011}
\item Wen Yu Mesh Generator \cite{SmithAndYu2009}
\item Sven Barendt Filtered Backprojection
\item Sanchez sDFT
\end{itemize}

\end{frame}

%%%%%%%%%%%%%%%%%%%%%%%%%%%%%%%%%%%%%%%%%%%%%%%%%%%%%%

\begin{frame}[allowframebreaks]
\frametitle{References}

\bibliography{../../ReferenceMaterial/References}

\end{frame}

%%%%%%%%%%%%%%%%%%%%%%%%%%%%%%%%%%%%%%%%%%%%%%%%%%%%%%

\end{document}